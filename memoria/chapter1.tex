\cleardoublepage

\chapter{Introducción}
\label{makereference}

Este proyecto tiene como \textbf{objetivo} predecir, a través de herramientas informáticas, \textbf{la radiación solar} en un punto teniendo en cuenta factores geográficos y meteorológicos.

Se quiere medir la irradiación solar ya que a partir de esta variable y un modelo físico del panel solar, se puede estimar cuanta energía se va a generar. 

Ya existen modelos teóricos para saber cual va a ser \textbf{la irradiación solar} en un lugar, en función por ejemplo de si no hubiese nubes o algunas otras magnitudes. 

Algunas de las\textbf{ magnitudes} que se puede utilizar para hacer nuestra estimación son:

La \textbf{radiación solar} varía debido a diversos factores, como los efectos atmosféricos, las variaciones locales en la atmósfera como el vapor de agua, las nubes y la contaminación, otros como la latitud y la estación del año y la hora del día.

Por ejemplo, la \textbf{latitud} influye en la cantidad de radiación solar que llega a la superficie; por otro lado, la radiación disminuirá cuanta más cantidad de nubes y más alta sea la \textbf{humedad}. Al contrario pasará con la \textbf{temperatura}, cuanto más altas sean las temperaturas mayor será la radiación. Otro factor que influye a la hora de medir la radiación, es el tipo de \textbf{superficie}, porque la reflexión de los rayos varía según el tipo. En la nieve se refleja un 85 por ciento, al contrario del asfalto que solo un 2 por ciento.

Existen otras variables que están también relacionadas con la irradiación y con la eficiencia de los paneles energéticos. Estas variables son por ejemplo, \textbf{la temperatura y la humedad}, las cuales se han usado en este proyecto, debido a que son las que consideramos \textbf{más relevantes} y de las que hablaremos en el siguiente capitulo, de manera más detallada.



\begin{figure}[htb]%t=top, b=bottom, h=here
	
	\begin{center}
		\includegraphics[height=2.5in]{figures/verano2015.png}
		\caption{Modelo verano 2015}
	\end{center}
    
    \label{figure1}
\end{figure}

\section{Motivación}
\label{makereference1.1}

Hoy en día es cada vez más importante el uso de \textbf{las energías renovables} y así utilizar cada vez menos los combustibles fósiles.
Además de la abundancia de estas energías y su gran aprovechamiento, la causa principal es que estas no producen gases de efecto invernadero que son los causantes del \textbf{cambio climático}, ni tampoco producen emisiones contaminantes. 

Adicionalmente de la ventaja más importante actualmente, que es su ayuda en contra del cambio climático, son inagotables o no generan residuos difíciles de tratar. 
Aunque también tienen algunos inconvenientes como por ejemplo, el gran impacto visual que tienen y las grandes cantidades de terreno que se necesitan para poder conseguir una cantidad significante de energía. Además no siempre se obtiene la misma cantidad de energía. Depende, por ejemplo, de la cantidad de sol o de viento en ese momento.

Este último inconveniente es el que hace que las compañías productoras y distribuidoras sean un poco reacias al uso de estas energías, debido a la incertidumbre de no saber de antemano \textbf{cuánto se va a producir}.

Es necesario una predicción de energía solar \textbf{precisa}, con distintos intervalos de tiempo, ya que cada vez es mayor el crecimiento del uso de las \textbf{energías renovables}. 
Es verdad, que debido a algunos factores como las nubes o la contaminación, predecir la energía solar es \textbf{más difícil}.
Pero esta predicción es realmente necesaria para poder tomar decisiones por parte de las empresas, para desarrollar estrategias de necesidad de comprar a otras y adecuar su producción y capacidad de reserva.

Se podría decir que todavía no existen modelos \textbf{realmente buenos} para predecir la energía solar. Hay algunas empresas como \href{https://aleasoft.com/es/}{Aleasoft}, que realizan estudios predictivos obteniendo previsiones horarias en tiempo real desde un día hasta diez días, pero sería necesario conocer la producción con un intervalo \textbf{de una o dos horas.}

\section{Breve descripción del sistema}
\label{makereference1.2}

Este sistema cuenta con cuatro grandes módulos de trabajo para llevar a cabo su función: nodo, servidor de datos, servidor de resultados y visualizador de datos.

\subsection{Nodo}
\label{makereference1.2.1}
Encargado de recoger los datos necesarios, está formado por distintos sensores que recogen la información meteorológica necesaria y una pequeña placa programable encargada de enviar esta información al \textbf{servidor de datos}.

\subsection{Servidor de datos}
\label{makereference1.2.2}
El \textbf{servidor de datos}, es el encargado de recibir, almacenar y distribuir la información obtenida por el \textbf{nodo}.

\subsection{Servidor de cálculo}
\label{makereference1.2.3}
Tiene como función recoger la información almacenada del servidor de datos, procesarla para obtener la predicción y enviarla al \textbf{visualizador de datos}. Puede estar, o no, en la misma máquina que el servidor de datos.

\subsection{Sistema de visualización}
\label{makereference1.2.4}
El sistema de visualización de datos elegido es ThingSpeak, una aplicación de código abierto que permite llevar un registro de los datos que se desean. Permite representarlos y estudiarlos.

\begin{figure}[htb]
    \begin{center}
        \includegraphics[height=2in]{figures/diagrama-sistema.png}
        \caption{Diagrama de los componentes del sistema con sus protocolos de comunicación.}
    \end{center}
    \label{diagrama-sistema}
\end{figure}
