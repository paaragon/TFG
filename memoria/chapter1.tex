\cleardoublepage

\chapter{Introducción}
%\label{ch:chapter1}
\label{makereference}

Aquí haremos un breve resumen de la motivación y de la descripción del sistema.

\begin{figure}[htb]%t=top, b=bottom, h=here
	
	\begin{center}
		\includegraphics[height=2.5in]{figures/verano2015.png}
		\caption{Modelo verano 2015}
	\end{center}
    
    \label{figure1}
\end{figure}

% +--------------------------------------------------------------------+
% |To create cross-references to figures, tables and segments
% |of text, LaTeX provides the following commands:
% |   \label{marker}
% |   \ref{marker}
% |   \pageref{marker}
% | where {marker} is a unique identifier.
% |
% | In the line above, we use \label{figure1} to mark a location
% | we wish to refer to later.  LATEX replaces \ref by the number of
% | the chapter, section, subsection, figure, or table after which the
% | corresponding \label command was issued. \pageref prints the page
% | number of the page where the \label command occurred.
% |
% +--------------------------------------------------------------------+

Here is an example of a Table:

\begin{table}

% +--------------------------------------------------------------------+
% | We include the command \begin{center} to center the table
% | horizontally on the page.  Note use of the command \end{center}
% | to turn off centering after the table is defined.
% +--------------------------------------------------------------------+
    \begin{center}

% +--------------------------------------------------------------------+
% | The table is created with this command
% |
% | \begin{tabular}[pos]{table spec}
% |
% | The "pos" argument specifies the vertical position of the table relative to
% | the baseline of the surrounding text.  Use t, b, or c to specify alignment
% | at the top, bottom, or center.
% |
% | The "table spec" command defines the format of the table
% |   l for a column of left-aligned text
% |   r for a column of right-aligned text
% |   c for centered text
% |   p{width} for a column containing justified text with line breaks
% |   | for a vertical line
% +--------------------------------------------------------------------+

    \begin{tabular}[c]{|c|c|c|}
        \hline
        Column 1 Heading & Column 2 Heading & Column 3 Heading \\
        \hline
        Col 1 Row 1 & Col 2 Row 1 & Col 3 Row 1\\
        Col 1 Row 2 & Col 2 Row 2 & Col 3 Row 2\\
        Col 1 Row 3 & Col 2 Row 3 & Col 3 Row 3\\
        \hline
    \end{tabular}
    \caption{Caption to appear below the table}
    \label{table1}
   \end{center}
\end{table}

\section{Motivación}
\label{makereference1.1}

In this paragraph, we want to refer to Fig.~\ref{figure1}
mentioned at the beginning of this chapter.  We also refer to the
Table~\ref{table1}.

\section{Breve descripción del sistema}
\label{makereference1.2}

In this section, we refer back to text mentioned in
Section~\ref{makereference1.1} on page~\pageref{makereference1.1}.

\section{Making a Citation}
\label{makereference1.3}

Here's an example of a citation to a single
work.~\cite{CT:Weiner:1999} It's also possible to make multiple
citations.~\cite{CT:Phillips:1985, ARP:Loy:1974}
