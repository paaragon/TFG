\cleardoublepage

\chapter{Nodo}
\label{makereference4}

\section{Introducción}
\label{makereference4.1}
Como hemos comentado anteriormente el \textbf{nodo} es quien recoge los datos meteorológicos escogidos y los envía al \textbf{servidor} de datos.

Se ha implementado con una \textbf{Raspberry Pi modelo 3} como placa programable y los sensores DHT22 para la medición de la temperatura y humedad y un piranómetro SP-212 para medir la irradiación solar. Ver imagen \ref{node-diagram}

La parte software del nodo ha sido implementada en \textbf{Python}.

\section{Elementos hardware utilizados}
\label{makereference4.2}

\subsection*{Raspberry Pi modelo 3}

Raspberry Pi  es un "pequeño ordenador", un computador de placa reducida \ref{rasp}. Con él se pueden controlar distintos componentes electrónicos a través de los pines que expone y combinarlo con la potencia y versatilidad de un sistema operativo.

\begin{figure}[htb]
	\begin{center}
		\includegraphics[height=5cm]{figures/Raspberry_Pi.jpg}
		\caption{Raspberry Pi Modelo 2b}
	\end{center}
	
	\label{rasp}
\end{figure}

Raspberry proporciona y sistema operativo oficial llamado Raspbian, una distribución de Debian. También soporta otros sistemas operativos.
 
Fue creado en 2006, en la Universidad de Cambridge, con el fin de fomentar la enseñanza en las escuela de ciencias de la computación, pero hasta 2012 no salió al mercado.

Existen distintos modelos de Raspberry Pi, desde la Raspberry Pi 1 Modelo A hasta la Raspberry Pi 3 Modelo B \ref{types}.
En nuestro proyecto hemos usado una Raspberry Pi 3 modelo B, con su sistema operativo oficial, Raspbian.

Se ha decidido utilizar este microcontrolador en el proyecto debido a su facilidad de manejo gracias a que cuenta con un sistema operativo y por ser relativamente barato para realizar un prototipo. Además cuenta con un gran comunidad detrás que trabaja en el desarrollo de nuevas guías y librerías para facilitar el desarrollo. 

\begin{figure}[htb]
	\begin{center}
		\includegraphics[width=17cm,height=12cm]{figures/Cuadro_Tipos_Raspberry.png}
		\caption{Cuadro comparativo de las especificaciones técnicas}
	\end{center}

	\label{types}
\end{figure}

\subsection*{Sensor de temperatura y humedad}

A la Raspberry se le conectan dos sensores, uno de temperatura y humedad y un piranómetro del cual hablaremos más tarde. Ambos para recoger los datos meteorológicos que necesitamos para nuestro modelo.

Para medir la temperatura y la humedad hemos utilizado un solo componente que recoge ambas muestras. El modelo escogido ha sido el sensor \textbf{DHT22}.

Algunas de las características de este sensor y en especial de este modelo (ya que tambien existe el modelo DHT11 de la misma marca) son que tiene una precisión de 0.5ºC para medir la temperatura, y entre un 2 y un 5 por ciento para la humedad. Ademas recogen dos muestras por segundo. Este sensor no es un sensor de alta precisión, pero es suficiente para nuestro proyecto, además de tener un precio muy económico. Ver imágen \ref{sensor}.

La comunicación entre el sensor y la Raspberry se realiza a través de un pin \textbf{GPIO} (General Purpose Input/Output, Entrada/Salida de Propósito General) que es un pin de propósito general. Estos pines pueden ser de entrada o de salida, y reciben y envían valores binarios de un bit.

En el \href{https://cdn-shop.adafruit.com/datasheets/Digital+humidity+and+temperature+sensor+AM2302.pdf}{datasheet} del sensor viene definido el proceso de comunicación entre este y la Raspberry. Para ello, el microcontrolador (Raspberry en nuestro caso) envía una señal de arranque para que el sensor cambie del estado "standby" al estado "running". Al terminar de enviar dicha señal, el sensor envía una respuesta de 40 bits que contiene la información relativa a la humedad (16 bits), a la temperatura (16 bits) y un checksum (8 bits) para comprobar que se ha enviado correctamente. Ver imagen \ref{DHT22comunication}.

El envío de estos bits sucede de la siguiente forma: el sensor envía una señal "baja" ("0") que indica el inicio de la transmisión de un bit. Seguidamente envía una señal " alta" ("1"). Si la transmisión de esta corriente dura más de 50\mu s indica que se ha transmitido un 1, en caso contrario, se quiso transmitir un 0.

\begin{figure}[htb]
	
	\begin{center}
		\includegraphics[width=15cm]{figures/DTH22comunicationdiagram.png}
		\caption{Diagrama de comunicación del sensor DHT22}
	\end{center}
	
	\label{DHT22comunication}
\end{figure} 

 \begin{figure}[htb]
	
	\begin{center}
		\includegraphics[width=7cm,height=7cm]{figures/sensorTemperaturaHumedad.png}
		\caption{Sensores de temperatura y humedad DHT11 y DHT22}
	\end{center}
	
	\label{sensor}
\end{figure} 

\subsection*{Piranómetro}

\section{Organización del código}
\label{makereference4.3} 
	\subsection{Librerías usadas}
		\subsubsection{Adafruit}
		\textbf{Adafruit} es una compañía de hardware open-source, que además de proporcionar \textbf{hardware}, suministra de una gran cantidad de documentación y librerías para facilitar el trabajo con sus componentes.
		
		\subsubsection{Pandas}
		\textbf{Pandas} es una biblioteca de software escrita para \textbf{Python} para la manipulación y análisis de datos. En particular, ofrece estructuras de datos y operaciones para manipular tablas numéricas y series temporales.
		
		\subsubsection{Paho}
		El proyecto \textbf{Paho} ha sido creado para proporcionar implementaciones escalables de código abierto de \textbf{protocolos de mensajería} abiertos y estándar dirigidos a aplicaciones nuevas, existentes y emergentes para Machine to Machine (M2M) e Internet of Things (IoT).
		
		Paho refleja las restricciones inherentes físicas y de costo de la conectividad del dispositivo. Los objetivos incluyen niveles efectivos de desacoplamiento entre dispositivos y aplicaciones, diseñados para mantener los mercados abiertos y fomentar el rápido crecimiento de middleware y aplicaciones escalables de Web y Enterprise. Paho inicialmente comenzó con implementaciones de cliente de publicación / suscripción de MQTT para su uso en plataformas incrustadas, y en el futuro traerá el soporte de servidor correspondiente según lo determinado por la comunidad.
		
	\subsection{Flujo de datos}

--- http://www.pveducation.org/pvcdrom/2-properties-sunlight/solar-radiation-earths-surface

La \textbf{radiación solar} varía debido a diversos factores, como los efectos atmosféricos, las variaciones locales en la atmósfera como el vapor de agua, las nubes y la contaminación, otros como la latitud y la estación del año y la hora del día.

--- https://www.imn.ac.cr/documents/10179/27818/factores-influyen-radiac-UV.pdf/187e5ea7-7c11-4ed7-955b-4e35c2f0ebf1

Por ejemplo, la \textbf{latitud} influye en la cantidad de radiación solar que llega a la superficie; por otro lado, la radiación disminuirá cuanta más cantidad de nubes y más alta sea la \textbf{humedad}. Al contrario pasará con la \textbf{temperatura}, cuanto más altas sean las temperaturas mayor será la radiación. Otro factor que influye a la hora de medir la radiación, es el tipo de \textbf{superficie}, porque la reflexión de los rayos varía según el tipo de superficie, la nieve se refleja un 85 por ciento, al contrario del asfalto que solo un 2 por ciento.

Hay otros elementos que podríamos haber medido y que no lo hemos hecho, como los comentados anteriormente o, por ejemplo también, la \textbf{velocidad del viento}, que nos importaría si refrigerara el panel.

Como hemos dicho, la superficie y la altitud son otros de estos elementos que están relacionados con la temperatura. Ya que por ejemplo cuando hemos sacado el nodo al exterior en el tejado para hacer las pruebas, rápidamente es alcanzado por el sol y alcanza altas temperaturas. No sería el mismo caso de estar en la superficie.

Por todo esto, lo que decidimos fue limitarnos a recoger información de radiación y temperatura porque son las más relevantes para nuestro modelo.

\begin{figure}[htb]
	
	\begin{center}
		\includegraphics[width=15cm,height=15cm]{figures/solar_project_node_diagram.png}
		\caption{Diagrama del nodo}
	\end{center}
	
	\label{node-diagram}
\end{figure}