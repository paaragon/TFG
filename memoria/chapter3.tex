\cleardoublepage

\chapter{Exploración del modelo predictivo}
\label{makereference3}

Here are more examples of referring to previous sections.  In
Chapter~\ref{makereference} there were several sections, including
section~\ref{makereference1.1}, section~\ref{makereference1.2},
and section~\ref{makereference1.3}.

Likewise, in Chapter~\ref{makereference2}, there are
sections~\ref{makereference2.1} and ~\ref{makereference2.2}.

\section{Introducción a los modelos predictivos}
\label{makereference3.1}

\section{Descripción de las librerías usadas}
\label{makereference3.2}
	\subsection{Scikit-learn}
	Scikit-learn es una biblioteca de aprendizaje de software libre para el lenguaje de programación Python. [2] Cuenta con varios algoritmos de clasificación, regresión y agrupación, incluyendo máquinas de vector de apoyo, bosques aleatorios, aumento de gradiente, k-medios y DBSCAN, y está diseñado para interoperar con las bibliotecas numéricas y científicas Python NumPy y SciPy.
	
	\subsection{NumPy}
	NumPy es una extensión de Python, que le agrega mayor soporte para vectores y matrices, constituyendo una biblioteca de funciones matemáticas de alto nivel para operar con esos vectores o matrices.
	
	\subsection{SciPy}
	SciPy es una biblioteca open source de herramientas y algoritmos matemáticos para Python. SciPy contiene módulos para optimización, álgebra lineal, integración, interpolación, funciones especiales, FFT, procesamiento de señales y de imagen, resolución de ODEs y otras tareas para la ciencia e ingeniería.
	
\section{Algoritmos utilizados}
\label{makereference3.3}
	\subsection{Regresión}
	\subsection{Clasificación}
	\subsection{Redes Neuronales}
	
\section{Protocolo del estudio}
\label{makereference3.4}

\section{Modelo escogido}
\label{makereference3.5}