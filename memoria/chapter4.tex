\cleardoublepage

\chapter{Nodo}
\label{makereference4}

Here are more examples of referring to previous sections.  In
Chapter~\ref{makereference} there were several sections, including
section~\ref{makereference1.1}, section~\ref{makereference1.2},
and section~\ref{makereference1.3}.

Likewise, in Chapter~\ref{makereference2}, there are
sections~\ref{makereference2.1} and ~\ref{makereference2.2}.

\section{Introducción}
\label{makereference4.1}

\section{Elementos utilizados}
\label{makereference4.2}
\subsection{Raspberry}
	\subsection{Sensor de temperatura y humedad}
	\subsection{Piranómetro}

\section{Organización del código}
\label{makereference4.3} 
	\subsection{Librerías usadas}
		\subsubsection{Adafruit}
		Adafruit es una compañía de hardware open-source, que además de proporcionar hardware, suministra de una gran cantidad de documentación y librerías para facilitar el trabajo con sus componentes.
		
		\subsubsection{Pandas}
		Pandas es una biblioteca de software escrita para Python para la manipulación y análisis de datos. En particular, ofrece estructuras de datos y operaciones para manipular tablas numéricas y series temporales.
		
		\subsubsection{Paho}
		El proyecto Paho ha sido creado para proporcionar implementaciones escalables de código abierto de protocolos de mensajería abiertos y estándar dirigidos a aplicaciones nuevas, existentes y emergentes para Machine to Machine (M2M) e Internet of Things (IoT).
		
		Paho refleja las restricciones inherentes físicas y de costo de la conectividad del dispositivo. Los objetivos incluyen niveles efectivos de desacoplamiento entre dispositivos y aplicaciones, diseñados para mantener los mercados abiertos y fomentar el rápido crecimiento de middleware y aplicaciones escalables de Web y Enterprise. Paho inicialmente comenzó con implementaciones de cliente de publicación / suscripción de MQTT para su uso en plataformas incrustadas, y en el futuro traerá el soporte de servidor correspondiente según lo determinado por la comunidad.
		
	\subsection{Flujo de datos}

\section{Método de instalación y puesta en marcha}
\label{makereference4.4}