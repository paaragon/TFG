\cleardoublepage

\chapter{Instalación y puesta en marcha}
\label{makereference5}

\section{Nodo}
\label{makereference5.1}
\subsection{Prerrequisitos}
\label{makereference5.1.2}
\subsubsection{Hardware}
\label{makereference5.1.3}
	\begin{itemize}
		\item \textbf{Raspberry Pi 2 model B} con una distribución Linux (nosotros usamos Raspbian)
		\item 1 Led
			\begin{itemize}
				\item Ánodo a 220 0hms de resistencia y la resistencia a vcc (3.3V)
				\item Cátodo a GPIO27
			\end{itemize}
		\item \textbf{Sensor DHT22} para medir humedad y temperatura
			\begin{itemize}
				\item Pin 1 a vcc (3.3V)
				\item Pin 2 a vcc (3.3V) con 10K 0hms de resistencia entre él y GPIO04
				\item Pin 3 desconectado
				\item Pin 4 a GND
			\end{itemize}
		\item \textbf{Piranómetro}
	\end{itemize}

	Las conexiones vienen explicadas en:
\lstset{language=bash}
\begin{lstlisting}[frame=single]
$ node/node_setup.txt
\end{lstlisting}

\subsubsection{Software}
	En la Raspberry Pi necesitamos
	\begin{itemize}
		\item \textbf{paho-mqtt}
	\end{itemize}
	Ejemplo de nuestro cliente MQTT
\lstset{language=python}
\begin{lstlisting}[frame=single]
import paho.mqtt.client as mqtt
import json

def listen_on_connect(client, userdata, rc):
print "Connected with result code "+ str(rc)
# Subscribing in on_connect() means that if we lose the connection
# and reconnect then subscriptions will be renewed.
client.subscribe("solar")

def listen_on_message(client, userdata, msg):
print msg.topic+" "+ str(msg.payload)

def sendToBroker(brokerIp, brokerPort, payload, topic):

client = mqtt.Client()
client.connect(brokerIp, brokerPort, 60)

result = client.publish(topic, payload=payload)
if result[0] == mqtt.MQTT_ERR_NO_CONN:
return False

client.disconnect()
return True

def listenToBroker(brokerIp, brokerPort, topic):

client = mqtt.Client()
client.on_connect = listen_on_connect
client.on_message = listen_on_message

client.connect(brokerIp, brokerPort, 60)

client.loop_forever()
\end{lstlisting}

\lstset{language=bash}
\begin{lstlisting}[frame=single]
$ pip install paho-mqtt
\end{lstlisting}

\subsection{Instalación}
\label{makereference5.1.4}
	\begin{itemize}
		\item Introduce el directiorio \textbf{node} en tu Rapsberry Pi
		\item Cambia las variables \textbf{brokerIp, brokerPort, topic, ubication} en solar-node.py
	\end{itemize}

\subsection{Uso}
\label{makereference5.1.5}
Ejecutar:
\begin{lstlisting}[frame=single]
$ node/solar_node.py
\end{lstlisting}

\section{Servidor de datos}
\label{makereference5.2}
\subsection{Requisitos previos}
\begin{itemize}
\item Máquina con distribucion Linux instalada. Recomendable Debian, Fedora, OpenSUSE o Ubuntu.
\item conexión a internet.
\item IP fija y opcionalmente, nombre de dominio.
\end{itemize}

La instalación de un broker MQTT es muy sencilla (una de sus principales ventajas). Para ello basta con instalar el demonio mosquitto a través del siguiente comando:

Ejemplo de instalación en Ubuntu:
\lstset{language=bash}
\begin{lstlisting}[frame=single]
$ sudo apt-get install mosquitto
\end{lstlisting}

Por defecto, Ubuntu arranca el servicio después de instalarlo. Si la distribución Lunix sobre la cual se instala no realiza esta acción por defecto, se deberá configurar debidamente para arrancarlo.

Hablar aquí de como securizar ubuntu y cambiar la configuración por defecto.
Aquí más info: https://www.digitalocean.com/community/tutorials/how-to-install-and-secure-the-mosquitto-mqtt-messaging-broker-on-ubuntu-16-04